% Manual of pgf-umlcd.sty, a convenient set of macros for drawing UML
% class diagrams.
% Written by Xu Yuan <xuyuan.cn@gmail.com>
% This file is part of pgf-umlcd
% you may get it at http://code.google.com/p/pgf-umlcd/

\documentclass{article}
\usepackage[margin=12mm]{geometry}
\usepackage{hyperref}

\usepackage[
% school,
% simplified
]{pgf-umlcd}

%%%%%%%%%%%%%%%%%%%%%%%%%%%%%%%%%%%%%%%%%%%%%%%%%%%%%%%%%%%%%%%%%
\usepackage{listings}
\usepackage{color}
\definecolor{listinggray}{gray}{0.92}
\lstset{ %
language=[LaTeX]TeX,
breaklines=true,
frame=single,
% frameround=tttt,
basicstyle=\footnotesize\ttfamily,
backgroundcolor=\color{listinggray},
keywordstyle=\color{blue}
}
%%%%%%%%%%%%%%%%%%%%%%%%%%%%%%%%%%%%%%%%%%%%%%%%%%%%%%%%%%%%%%%%%

%%%%%%%%%%%%%%%%%%%%%%%%%%%%%%%%%%%%%%%%%%%%%%%%%%%%%%%%%%%%%%%%%
\hypersetup{
  colorlinks=true,
  linkcolor=blue,
  anchorcolor=black,
  citecolor=olive,
  filecolor=magenta,
  menucolor=red,
  urlcolor=blue
}
%%%%%%%%%%%%%%%%%%%%%%%%%%%%%%%%%%%%%%%%%%%%%%%%%%%%%%%%%%%%%%%%%

%%%%%%%%%%%%%%%%%%%%%%%%%%%%%%%%%%%%%%%%%%%%%%%%%%%%%%%%%%%%%%%%%
\newcommand{\demo}[2][1]{
\begin{minipage}{.49\linewidth}
\centering
\resizebox{#1\linewidth}{!}{
\input{demo/#2}
}
\end{minipage}
\hspace{0.01\linewidth}
\begin{minipage}{.5\linewidth}
\lstinputlisting{demo/#2}
\end{minipage}
}
%%%%%%%%%%%%%%%%%%%%%%%%%%%%%%%%%%%%%%%%%%%%%%%%%%%%%%%%%%%%%%%%%

%%%%%%%%%%%%%%%%%%%%%%%%%%%%%%%%%%%%%%%%%%%%%%%%%%%%%%%%%%%%%%%%%
\newcommand{\example}[1]{
\resizebox{\linewidth}{!}{
\input{demo/#1}
}
\lstinputlisting{demo/#1}
}
%%%%%%%%%%%%%%%%%%%%%%%%%%%%%%%%%%%%%%%%%%%%%%%%%%%%%%%%%%%%%%%%%

\begin{document}
%%%%%%%%%%%%%%%%%%%%%%%%%%%%%%%%%%%%%%%%%%%%%%%%%%%%%%%%%%%%%%%%%
\title{Drawing UML Class Diagram by using \texttt{pgf-umlcd}}
\author{\href{mailto:xuyuan.cn@gmail.com}{Yuan Xu}}
\date{\today{}~(v0.3)}
\maketitle
%%%%%%%%%%%%%%%%%%%%%%%%%%%%%%%%%%%%%%%%%%%%%%%%%%%%%%%%%%%%%%%%%

\begin{abstract}
  \texttt{pgf-umlcd} is a LaTeX package for drawing UML Class
  Diagrams. As stated by its name, it is based on a very popular
  graphic package \texttt{PGF/TikZ}. This document presents the usage
  of \texttt{pgf-umlcd} and collects some UML class diagrams as
  examples. \texttt{pgf-umlcd} can be downloaded from
  \href{http://code.google.com/p/pgf-umlcd/}{http://code.google.com/p/pgf-umlcd/}.
\end{abstract}

\tableofcontents

\section{Basics}
\subsection{Class with attributes and operations}
Note: If you don't want to show empty parts in the diagrams, please
use \texttt{simplified} option, e.g. \lstinline|\usepackage[simplified]{pgf-umlcd}|.\\
\demo{class}

\subsubsection{Visibility of attributes and operations}
\demo[0.8]{visibility}

\subsubsection{Abstract class and interface}
\demo[0.5]{abstract-class}
\demo[0.5]{interface}

\subsubsection{Object}
\demo[0.7]{object}
Note: Object with rounded corners and methods are used in German school for didactic reasons. You get the rounded corners with \lstinline|\usepackage[school]{pgf-umlcd}|. If you need both in one document you can switch it with \lstinline|\switchUmlcdSchool| \\
\switchUmlcdSchool
\demo[0.7]{object}
\demo[0.7]{object-include-methods}

\subsubsection{Note}
The \lstinline|\umlnote| use the same syntax as tikz command
\lstinline|\node|, e.g. \lstinline|\umlnote[style] (name) at (coordinate) {text};|

\demo[0.7]{note}

\subsection{Inheritance and implement}
\subsubsection{Inheritance}
\demo{inheritance}
\subsubsection{Multiple Inheritance}
\demo{multiple-inheritance}

\subsubsection{Implement an interface}
\demo{implement-interface}

\subsection{Association, Aggregation and Composition}
\subsubsection{Association}
\demo{association}

\subsubsection{Unidirectional association}
\demo{unidirectional-association}

\subsubsection{Aggregation}
\demo{aggregation}

\subsubsection{Composition}
\demo{composition}

\subsection{Package}
\demo{package}

\section{Customization}
\subsection{Color settings}
The color of digram is defined by \lstinline|\umltextcolor|, \lstinline|\umldrawcolor| and \lstinline|\umlfillcolor|, such as:

\demo{color}


\section{Examples}
\subsection{Abstract Factory}
\example{abstract-factory}

\section{Acknowledgements}
Many people contributed to \texttt{pgf-umlcd} by reporting problems,
suggesting various improvements or submitting code. Here is a list of
these people: \href{mailto:martin.quinson@loria.fr}{Martin Quinson},
\href{mailto:johannes_pieper@yahoo.de}{Johannes Pieper},
\href{mailto:lwghsh@gmail.com}{sh w},
and \href{mailto:maartenvandessel@hotmail.com}{Maarten van Dessel}.

\end{document}
%%% Local Variables:
%%% mode: Tex-PDF
%%% TeX-master: t
%%% End:
